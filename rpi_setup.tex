% Options for packages loaded elsewhere
\PassOptionsToPackage{unicode}{hyperref}
\PassOptionsToPackage{hyphens}{url}
%
\documentclass[
]{article}
\usepackage{amsmath,amssymb}
\usepackage{iftex}
\ifPDFTeX
  \usepackage[T1]{fontenc}
  \usepackage[utf8]{inputenc}
  \usepackage{textcomp} % provide euro and other symbols
\else % if luatex or xetex
  \usepackage{unicode-math} % this also loads fontspec
  \defaultfontfeatures{Scale=MatchLowercase}
  \defaultfontfeatures[\rmfamily]{Ligatures=TeX,Scale=1}
\fi
\usepackage{lmodern}
\ifPDFTeX\else
  % xetex/luatex font selection
\fi
% Use upquote if available, for straight quotes in verbatim environments
\IfFileExists{upquote.sty}{\usepackage{upquote}}{}
\IfFileExists{microtype.sty}{% use microtype if available
  \usepackage[]{microtype}
  \UseMicrotypeSet[protrusion]{basicmath} % disable protrusion for tt fonts
}{}
\makeatletter
\@ifundefined{KOMAClassName}{% if non-KOMA class
  \IfFileExists{parskip.sty}{%
    \usepackage{parskip}
  }{% else
    \setlength{\parindent}{0pt}
    \setlength{\parskip}{6pt plus 2pt minus 1pt}}
}{% if KOMA class
  \KOMAoptions{parskip=half}}
\makeatother
\usepackage{xcolor}
\usepackage[margin=1in]{geometry}
\usepackage{color}
\usepackage{fancyvrb}
\newcommand{\VerbBar}{|}
\newcommand{\VERB}{\Verb[commandchars=\\\{\}]}
\DefineVerbatimEnvironment{Highlighting}{Verbatim}{commandchars=\\\{\}}
% Add ',fontsize=\small' for more characters per line
\usepackage{framed}
\definecolor{shadecolor}{RGB}{248,248,248}
\newenvironment{Shaded}{\begin{snugshade}}{\end{snugshade}}
\newcommand{\AlertTok}[1]{\textcolor[rgb]{0.94,0.16,0.16}{#1}}
\newcommand{\AnnotationTok}[1]{\textcolor[rgb]{0.56,0.35,0.01}{\textbf{\textit{#1}}}}
\newcommand{\AttributeTok}[1]{\textcolor[rgb]{0.13,0.29,0.53}{#1}}
\newcommand{\BaseNTok}[1]{\textcolor[rgb]{0.00,0.00,0.81}{#1}}
\newcommand{\BuiltInTok}[1]{#1}
\newcommand{\CharTok}[1]{\textcolor[rgb]{0.31,0.60,0.02}{#1}}
\newcommand{\CommentTok}[1]{\textcolor[rgb]{0.56,0.35,0.01}{\textit{#1}}}
\newcommand{\CommentVarTok}[1]{\textcolor[rgb]{0.56,0.35,0.01}{\textbf{\textit{#1}}}}
\newcommand{\ConstantTok}[1]{\textcolor[rgb]{0.56,0.35,0.01}{#1}}
\newcommand{\ControlFlowTok}[1]{\textcolor[rgb]{0.13,0.29,0.53}{\textbf{#1}}}
\newcommand{\DataTypeTok}[1]{\textcolor[rgb]{0.13,0.29,0.53}{#1}}
\newcommand{\DecValTok}[1]{\textcolor[rgb]{0.00,0.00,0.81}{#1}}
\newcommand{\DocumentationTok}[1]{\textcolor[rgb]{0.56,0.35,0.01}{\textbf{\textit{#1}}}}
\newcommand{\ErrorTok}[1]{\textcolor[rgb]{0.64,0.00,0.00}{\textbf{#1}}}
\newcommand{\ExtensionTok}[1]{#1}
\newcommand{\FloatTok}[1]{\textcolor[rgb]{0.00,0.00,0.81}{#1}}
\newcommand{\FunctionTok}[1]{\textcolor[rgb]{0.13,0.29,0.53}{\textbf{#1}}}
\newcommand{\ImportTok}[1]{#1}
\newcommand{\InformationTok}[1]{\textcolor[rgb]{0.56,0.35,0.01}{\textbf{\textit{#1}}}}
\newcommand{\KeywordTok}[1]{\textcolor[rgb]{0.13,0.29,0.53}{\textbf{#1}}}
\newcommand{\NormalTok}[1]{#1}
\newcommand{\OperatorTok}[1]{\textcolor[rgb]{0.81,0.36,0.00}{\textbf{#1}}}
\newcommand{\OtherTok}[1]{\textcolor[rgb]{0.56,0.35,0.01}{#1}}
\newcommand{\PreprocessorTok}[1]{\textcolor[rgb]{0.56,0.35,0.01}{\textit{#1}}}
\newcommand{\RegionMarkerTok}[1]{#1}
\newcommand{\SpecialCharTok}[1]{\textcolor[rgb]{0.81,0.36,0.00}{\textbf{#1}}}
\newcommand{\SpecialStringTok}[1]{\textcolor[rgb]{0.31,0.60,0.02}{#1}}
\newcommand{\StringTok}[1]{\textcolor[rgb]{0.31,0.60,0.02}{#1}}
\newcommand{\VariableTok}[1]{\textcolor[rgb]{0.00,0.00,0.00}{#1}}
\newcommand{\VerbatimStringTok}[1]{\textcolor[rgb]{0.31,0.60,0.02}{#1}}
\newcommand{\WarningTok}[1]{\textcolor[rgb]{0.56,0.35,0.01}{\textbf{\textit{#1}}}}
\usepackage{graphicx}
\makeatletter
\def\maxwidth{\ifdim\Gin@nat@width>\linewidth\linewidth\else\Gin@nat@width\fi}
\def\maxheight{\ifdim\Gin@nat@height>\textheight\textheight\else\Gin@nat@height\fi}
\makeatother
% Scale images if necessary, so that they will not overflow the page
% margins by default, and it is still possible to overwrite the defaults
% using explicit options in \includegraphics[width, height, ...]{}
\setkeys{Gin}{width=\maxwidth,height=\maxheight,keepaspectratio}
% Set default figure placement to htbp
\makeatletter
\def\fps@figure{htbp}
\makeatother
\setlength{\emergencystretch}{3em} % prevent overfull lines
\providecommand{\tightlist}{%
  \setlength{\itemsep}{0pt}\setlength{\parskip}{0pt}}
\setcounter{secnumdepth}{-\maxdimen} % remove section numbering
\ifLuaTeX
  \usepackage{selnolig}  % disable illegal ligatures
\fi
\usepackage{bookmark}
\IfFileExists{xurl.sty}{\usepackage{xurl}}{} % add URL line breaks if available
\urlstyle{same}
\hypersetup{
  pdftitle={Particle Argon Setup},
  pdfauthor={Jake Hosen},
  hidelinks,
  pdfcreator={LaTeX via pandoc}}

\title{Particle Argon Setup}
\author{Jake Hosen}
\date{}

\begin{document}
\maketitle

\subsection{Stuff to install first}\label{stuff-to-install-first}

\section{Raspberry Pi NoIR Camera Timelapse Setup
Instructions}\label{raspberry-pi-noir-camera-timelapse-setup-instructions}

This document provides step-by-step instructions for setting up a Python
script that automatically takes photos using a Raspberry Pi NoIR camera
every 15 minutes and starts on boot.

\subsection{Getting your login setup}\label{getting-your-login-setup}

\begin{itemize}
\tightlist
\item
  Create a user called ``pi''.
\item
  Using the GUI (windows) interface, setup your wifi login.
\item
  Open a terminal and type \texttt{sudo\ raspi-config}.
\item
  When you open this menu, select ``1. System Options''. Then select
  ``B5 Boot / Auto Login''. Then Select ``B2 Console Autologin''.
\item
  Exit from raspi-config and reboot your system.
\end{itemize}

\subsection{Python Script
(timelapse.py)}\label{python-script-timelapse.py}

Create the Python script that will control the camera:

\begin{Shaded}
\begin{Highlighting}[]
\CommentTok{\#!/usr/bin/env python3}
\CommentTok{\# timelapse.py {-} Takes photos every 15 minutes using libcamera}

\ImportTok{import}\NormalTok{ os}
\ImportTok{import}\NormalTok{ time}
\ImportTok{import}\NormalTok{ datetime}
\ImportTok{import}\NormalTok{ subprocess}
\ImportTok{from}\NormalTok{ pathlib }\ImportTok{import}\NormalTok{ Path}

\CommentTok{\# Configuration}
\NormalTok{INTERVAL }\OperatorTok{=} \DecValTok{15} \OperatorTok{*} \DecValTok{60}  \CommentTok{\# 15 minutes in seconds}
\NormalTok{SAVE\_DIRECTORY }\OperatorTok{=} \StringTok{"/home/pi/timelapse"}
\NormalTok{IMAGE\_PREFIX }\OperatorTok{=} \StringTok{"timelapse\_"}

\KeywordTok{def}\NormalTok{ setup():}
    \CommentTok{"""Create the save directory if it doesn\textquotesingle{}t exist."""}
\NormalTok{    Path(SAVE\_DIRECTORY).mkdir(parents}\OperatorTok{=}\VariableTok{True}\NormalTok{, exist\_ok}\OperatorTok{=}\VariableTok{True}\NormalTok{)}
    \BuiltInTok{print}\NormalTok{(}\SpecialStringTok{f"Images will be saved to: }\SpecialCharTok{\{}\NormalTok{SAVE\_DIRECTORY}\SpecialCharTok{\}}\SpecialStringTok{"}\NormalTok{)}

\KeywordTok{def}\NormalTok{ take\_photo():}
    \CommentTok{"""Take a photo using libcamera{-}still and save it with timestamp."""}
\NormalTok{    timestamp }\OperatorTok{=}\NormalTok{ datetime.datetime.now().strftime(}\StringTok{"\%Y\%m}\SpecialCharTok{\%d}\StringTok{\_\%H\%M\%S"}\NormalTok{)}
\NormalTok{    filename }\OperatorTok{=} \SpecialStringTok{f"}\SpecialCharTok{\{}\NormalTok{IMAGE\_PREFIX}\SpecialCharTok{\}\{}\NormalTok{timestamp}\SpecialCharTok{\}}\SpecialStringTok{.jpg"}
\NormalTok{    filepath }\OperatorTok{=}\NormalTok{ os.path.join(SAVE\_DIRECTORY, filename)}
    
    \ControlFlowTok{try}\NormalTok{:}
        \CommentTok{\# Using libcamera{-}still command}
\NormalTok{        command }\OperatorTok{=}\NormalTok{ [}
            \StringTok{"libcamera{-}still"}\NormalTok{,}
            \StringTok{"{-}o"}\NormalTok{, filepath,}
            \StringTok{"{-}{-}nopreview"}\NormalTok{,}
            \StringTok{"{-}{-}timeout"}\NormalTok{, }\StringTok{"1000"}  \CommentTok{\# 1 second timeout}
\NormalTok{        ]}
        
        \CommentTok{\# Execute the command}
\NormalTok{        subprocess.run(command, check}\OperatorTok{=}\VariableTok{True}\NormalTok{)}
        \BuiltInTok{print}\NormalTok{(}\SpecialStringTok{f"Photo taken: }\SpecialCharTok{\{}\NormalTok{filename}\SpecialCharTok{\}}\SpecialStringTok{"}\NormalTok{)}
        \ControlFlowTok{return} \VariableTok{True}
    \ControlFlowTok{except}\NormalTok{ subprocess.CalledProcessError }\ImportTok{as}\NormalTok{ e:}
        \BuiltInTok{print}\NormalTok{(}\SpecialStringTok{f"Error taking photo: }\SpecialCharTok{\{}\NormalTok{e}\SpecialCharTok{\}}\SpecialStringTok{"}\NormalTok{)}
        \ControlFlowTok{return} \VariableTok{False}
    \ControlFlowTok{except} \PreprocessorTok{Exception} \ImportTok{as}\NormalTok{ e:}
        \BuiltInTok{print}\NormalTok{(}\SpecialStringTok{f"Unexpected error: }\SpecialCharTok{\{}\NormalTok{e}\SpecialCharTok{\}}\SpecialStringTok{"}\NormalTok{)}
        \ControlFlowTok{return} \VariableTok{False}

\KeywordTok{def}\NormalTok{ main():}
    \CommentTok{"""Main function to run the timelapse."""}
\NormalTok{    setup()}
    \BuiltInTok{print}\NormalTok{(}\StringTok{"Starting timelapse capture..."}\NormalTok{)}
    
    \ControlFlowTok{try}\NormalTok{:}
        \ControlFlowTok{while} \VariableTok{True}\NormalTok{:}
            \CommentTok{\# Take photo}
\NormalTok{            success }\OperatorTok{=}\NormalTok{ take\_photo()}
            
            \CommentTok{\# Wait for the next interval}
            \BuiltInTok{print}\NormalTok{(}\SpecialStringTok{f"Waiting }\SpecialCharTok{\{}\NormalTok{INTERVAL}\SpecialCharTok{\}}\SpecialStringTok{ seconds until next capture..."}\NormalTok{)}
\NormalTok{            time.sleep(INTERVAL)}
    \ControlFlowTok{except} \PreprocessorTok{KeyboardInterrupt}\NormalTok{:}
        \BuiltInTok{print}\NormalTok{(}\StringTok{"Timelapse capture stopped by user."}\NormalTok{)}
    \ControlFlowTok{except} \PreprocessorTok{Exception} \ImportTok{as}\NormalTok{ e:}
        \BuiltInTok{print}\NormalTok{(}\SpecialStringTok{f"Error in main loop: }\SpecialCharTok{\{}\NormalTok{e}\SpecialCharTok{\}}\SpecialStringTok{"}\NormalTok{)}

\ControlFlowTok{if} \VariableTok{\_\_name\_\_} \OperatorTok{==} \StringTok{"\_\_main\_\_"}\NormalTok{:}
\NormalTok{    main()}
\end{Highlighting}
\end{Shaded}

\subsection{Systemd Service
(timelapse.service)}\label{systemd-service-timelapse.service}

Create a systemd service to run the script at boot:

\begin{Shaded}
\begin{Highlighting}[]
\KeywordTok{[Unit]}
\DataTypeTok{Description}\OtherTok{=}\StringTok{Raspberry Pi NoIR Camera Timelapse}
\DataTypeTok{After}\OtherTok{=}\StringTok{network.target}

\KeywordTok{[Service]}
\DataTypeTok{ExecStart}\OtherTok{=}\StringTok{/usr/bin/python3 /home/pi/timelapse.py}
\DataTypeTok{WorkingDirectory}\OtherTok{=}\StringTok{/home/pi}
\DataTypeTok{StandardOutput}\OtherTok{=}\StringTok{append:/home/pi/timelapse.log}
\DataTypeTok{StandardError}\OtherTok{=}\StringTok{append:/home/pi/timelapse.log}
\DataTypeTok{Restart}\OtherTok{=}\StringTok{always}
\DataTypeTok{User}\OtherTok{=}\StringTok{pi}

\KeywordTok{[Install]}
\DataTypeTok{WantedBy}\OtherTok{=}\StringTok{multi{-}user.target}
\end{Highlighting}
\end{Shaded}

\subsection{Installation Steps}\label{installation-steps}

\subsubsection{Step 1: Create the Python
script}\label{step-1-create-the-python-script}

\begin{enumerate}
\def\labelenumi{\arabic{enumi}.}
\item
  Plug-in a USB drive with the files you need and open a terminal on
  your Raspberry Pi
\item
  Create the Python script file:

\begin{Shaded}
\begin{Highlighting}[]
\FunctionTok{nano}\NormalTok{ /home/pi/timelapse.py}
\end{Highlighting}
\end{Shaded}
\item
  Copy and paste the Python script from above
\item
  Save and exit (Ctrl+X, then Y, then Enter)
\item
  Make the script executable:

\begin{Shaded}
\begin{Highlighting}[]
\FunctionTok{chmod}\NormalTok{ +x /home/pi/timelapse.py}
\end{Highlighting}
\end{Shaded}
\end{enumerate}

\subsubsection{Step 2: Create the systemd
service}\label{step-2-create-the-systemd-service}

\begin{enumerate}
\def\labelenumi{\arabic{enumi}.}
\item
  Create a systemd service file:

\begin{Shaded}
\begin{Highlighting}[]
\FunctionTok{sudo}\NormalTok{ nano /etc/systemd/system/timelapse.service}
\end{Highlighting}
\end{Shaded}
\item
  Copy and paste the service configuration from above
\item
  Save and exit (Ctrl+X, then Y, then Enter)
\item
  Make sure that you account is configured to access the files
  appropriately:
\end{enumerate}

\begin{Shaded}
\begin{Highlighting}[]
\FunctionTok{sudo}\NormalTok{ usermod }\AttributeTok{{-}a} \AttributeTok{{-}G}\NormalTok{ video,gpio pi}
\FunctionTok{sudo}\NormalTok{ usermod }\AttributeTok{{-}a} \AttributeTok{{-}G}\NormalTok{ i2c,spi pi}

\CommentTok{\#\#\# Step 3: Enable and start the service}
\KeywordTok{\textasciigrave{}\textasciigrave{}\textasciigrave{}}\FunctionTok{bash}
\FunctionTok{sudo}\NormalTok{ systemctl daemon{-}reload}
\FunctionTok{sudo}\NormalTok{ systemctl enable timelapse.service}
\FunctionTok{sudo}\NormalTok{ systemctl start timelapse.service}
\end{Highlighting}
\end{Shaded}

\subsubsection{Step 4: Check the status}\label{step-4-check-the-status}

\begin{Shaded}
\begin{Highlighting}[]
\FunctionTok{sudo}\NormalTok{ systemctl status timelapse.service}
\end{Highlighting}
\end{Shaded}

\subsection{Additional Information}\label{additional-information}

\begin{itemize}
\tightlist
\item
  Photos will be saved to \texttt{/home/pi/timelapse/} with filenames
  like \texttt{timelapse\_20250402\_153000.jpg}
\item
  Logs will be saved to \texttt{/home/pi/timelapse.log}
\item
  You can modify the \texttt{INTERVAL} variable in the script to change
  the time between photos
\item
  Make sure your Raspberry Pi NoIR camera is properly connected and
  enabled in \texttt{raspi-config}
\item
  The script uses \texttt{libcamera-still} which is standard for
  Raspberry Pi OS Bullseye and newer
\end{itemize}

\subsection{Managing the Service}\label{managing-the-service}

To check the service status:

\begin{Shaded}
\begin{Highlighting}[]
\FunctionTok{sudo}\NormalTok{ systemctl status timelapse.service}
\end{Highlighting}
\end{Shaded}

To stop the service:

\begin{Shaded}
\begin{Highlighting}[]
\FunctionTok{sudo}\NormalTok{ systemctl stop timelapse.service}
\end{Highlighting}
\end{Shaded}

To start the service:

\begin{Shaded}
\begin{Highlighting}[]
\FunctionTok{sudo}\NormalTok{ systemctl start timelapse.service}
\end{Highlighting}
\end{Shaded}

To view the logs:

\begin{Shaded}
\begin{Highlighting}[]
\FunctionTok{tail} \AttributeTok{{-}f}\NormalTok{ /home/pi/timelapse.log}
\end{Highlighting}
\end{Shaded}

\subsection{Troubleshooting}\label{troubleshooting}

If the camera isn't working: 1. Check that the camera is properly
connected 2. Ensure the camera is enabled in raspi-config:
\texttt{bash\ \ \ \ sudo\ raspi-config} Navigate to ``Interface
Options'' \textgreater{} ``Camera'' and enable it 3. Reboot the
Raspberry Pi: \texttt{bash\ \ \ \ sudo\ reboot}

If the service doesn't start: 1. Check the service status:
\texttt{bash\ \ \ \ sudo\ systemctl\ status\ timelapse.service} 2. Check
the logs: \texttt{bash\ \ \ \ tail\ -f\ /home/pi/timelapse.log}

\end{document}
